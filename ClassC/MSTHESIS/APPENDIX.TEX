.H 1 "Appendicies -- Sample ClassC Programs"

.H 2 ChessMovs.e  -- Chess Move Generator

.ls 1
.na
.nf
/********************************************************************
 *        ChessMovs.e -- Written in ClassC by Paul Kirkaas --
 *
 *        ChessMovs shows all legal chess moves for any given
 *        chess board configuration.
 *        The chess board and each piece are objects.  The pieces
 *        examine the board and the other pieces on it, and generate
 *        their own legal moves based on their information about themselves
 *        and the state of the board.
 **********************************************************************/

#include "List.h"
class Board
{  Piece square[8][8];  // array of Pieces
   display();
   setup();
   char  sect[8][8][9];
   object moves; // Set of moves -- gets set/initialized by the piece
   void add();
   delete();
};
class Pos { int x;
            int y;
            char * eval();
            object copy();
           };
class Piece
{
  object pos; /* position object, of class Pos */
  char colr;
  void showmoves();
  object genmoves(); /* Returns an object of class Set, of moves */
  char* eval();
  delete();
  init();
  char id();
 };
object Piece::genmoves() {}
char Piece::id() {}
class Rook : Piece
{
 char id();
 object genmoves(); /* Returns an object of class Set, of moves */
 object Vmoves();
 };
class Bishop : Piece
{char id();
 object genmoves();
 object DiagonalMoves();
 };


class Queen : Rook : Bishop

// An example of multiple inheritence -- the queen has characteristics of
// both the rook and bishop.

{char id() ;
 object genmoves();
 };

extern Board board;
Board::delete()
{ int i,j;
  for ( i=0; i < 8; i++) for ( j=0; j < 8; j++) square[i][j]:>delete();
  moves:>delete();
}


Board::init()
  {
   int i,j;
   for (i=0; i<8; i++) for (j=0;j<8;j++)square[i][j] = 0;
  }
void Board::add(pP, i, j) Piece pP; int i,j;
{square[i][j] = pP;}
Board::display()
{ setup();
  char line[99];
  strcpy(line,
        "_________________________________________________________________\n");
  printf("%s",line); // Go Home & draw 1st line.
  int i,j;
  for (j=0;j<8;j++)
     {
     for (i=0;i<8;i++) printf("|%s",sect[i][7-j]);
     printf("|\n%s",line);
     }
};

Board::setup()
  { int i,j;
    for(i=0;i<8;i++) for (j=0;j<8;j++)
      { strcpy (sect[i][j],"       ");
        if (square[i][j])
          {
            sect[i][j][1] = square[i][j]:>colr;
            sect[i][j][3] = (square[i][j]):>id();
          }
      }
    if (moves)
    {
    moves:>rewind();
    while (moves:>place)
       { sect[moves:>current():>(int)x][moves:>current():>(int)y][5] = 'X';
        moves:>next();
       }
    }
  };

/***********************************************************************/

char * Pos::eval()
{static char  PosEvalRet[2][99];
 static int ix = 0;
 ix = !ix;
 sprintf(PosEvalRet[ix],"Pos:x=%d; y=%d :: ",x,y);
 return PosEvalRet[ix];
}
object Pos::copy()
{ object retobj = Pos:>new();
  retobj:>(int)x = x;
  retobj:>(int)y = y;
  return retobj;
}
Piece::delete() { pos :> delete(); }
Piece::init(cl,x,y)char cl;int x; int y ;
  {
   pos = Pos:>new();
   colr = cl;
   extern object board;
   board:>add(self,x,y);
   pos:>(int)x=x;
   pos:>(int)y=y;
}

char* Piece:: eval()
    {char outS [2][99];
     static int ix = 0;
     ix = !ix;
     sprintf(outS[ix]," Piece:: %c [%d,%d]\n",colr,
                pos:>(int)x,pos:>(int)y);
     return outS[ix];
    }

void Piece::showmoves(){// HERE IS WHERE WE NEED GARBAGE COLLECTION
                    board :> moves  =genmoves();
                    board :> display();
                  };

char Rook::id() {return 'R';};

object Rook::genmoves()
{ return Vmoves();}

object Rook::Vmoves()
{ Pos tstpos;
  object pieceP=0;
  object moves = Set:>new();
  tstpos = pos:>copy();

// Rook movement in Positive x direction.

  for ( (tstpos:>x) = pos:>(int)x+1;
        (tstpos:>x) < 8; (tstpos:>x)++)
     { pieceP =(board:>square[tstpos:>x][tstpos:>y]);
      if (pieceP)
        { if (pieceP:>(char)colr != colr) moves:>add(tstpos:>copy());
          break;
        }
      moves:>add(tstpos:>copy());
     }

tstpos = pos:>copy();
// Rook movement in Negative x direction.

  for ( (tstpos:>x) = pos:>(int)x-1;
        (tstpos:>x) >= 0; (tstpos:>x)--)
     { pieceP =(board:>square[tstpos:>x][tstpos:>y]);
      if (pieceP)
        { if (pieceP:>(char)colr != colr) moves:>add(tstpos:>copy());
          break;
        }
      moves:>add(tstpos:>copy());
     }


// Rook movement in Positive Y direction.

tstpos = pos:>copy();
  for ( (tstpos:>y) = pos:>(int)y+1;
        (tstpos:>y) < 8; (tstpos:>y)++)
     { pieceP =(board:>square[tstpos:>x][tstpos:>y]);
      if (pieceP)
        { if (pieceP:>(char)colr != colr) moves:>add(tstpos:>copy());
          break;
        }
      moves:>add(tstpos:>copy());
     }

// Rook movement in Negative Y direction.
tstpos = pos:>copy();

  for ( (tstpos:>y) = pos:>(int)y-1;
        (tstpos:>y) >= 0; (tstpos:>y)--)
     { pieceP =(board:>square[tstpos:>x][tstpos:>y]);
      if (pieceP)
        { if (pieceP:>(char)colr != colr) moves:>add(tstpos:>copy());
          break;
        }
      moves:>add(tstpos:>copy());
     }

return moves;
}

/***********************************************************************/

object Bishop::genmoves()
{ return DiagonalMoves(); }

char Bishop::id() {return 'B';};

/***********************************************************************/

object Bishop::DiagonalMoves()
{ object tstpos;
  object pieceP;
  object moves = Set:>new();
  tstpos = pos:>copy();
/* Generate diagonal move for Bishop in Positive X & Positive Y */
  for ( (tstpos:>(int)x) = pos:>(int)x+1,(tstpos:>(int)y) = pos:>(int)y+1;
        ((tstpos:>(int)x) < 8) && ((tstpos:>(int)y) < 8);
        (tstpos:>(int)x)++, (tstpos:>(int)y)++)
     { pieceP =  board:>square[tstpos:>x][tstpos:>y];
      if (pieceP)
        { if (pieceP:>(char)colr != colr)  moves:>add(tstpos:>copy());
          break;
        }
      moves:>add(tstpos:>copy());
     }


tstpos = pos:>copy();
/* Generate diagonal move for Bishop in Negative X & Negative Y */

  for ( (tstpos:>(int)x) = pos:>(int)x-1,(tstpos:>(int)y) = pos:>(int)y-1;
        ((tstpos:>(int)x) >= 0) && ((tstpos:>(int)y) >= 0);
        (tstpos:>(int)x)--, (tstpos:>(int)y)--)
     { pieceP =  (board:>square[tstpos:>(int)x][tstpos:>(int)y]);
      if (pieceP)
        { if (pieceP:>(char)colr != colr)  moves:>add(tstpos:>copy());
          break;
        }
      moves:>add(tstpos:>copy());
     }


/* Generate diagonal move for Bishop in Negative X & Positive Y */
tstpos = pos:>copy();
  for ( (tstpos:>(int)x) = pos:>(int)x-1,(tstpos:>(int)y) = pos:>(int)y+1;
        ((tstpos:>(int)x) >= 0) && ((tstpos:>(int)y) < 8);
        (tstpos:>(int)x)--, (tstpos:>(int)y)++)
     { pieceP =  (board:>square[tstpos:>(int)x][tstpos:>(int)y]);
      if (pieceP)
        { if (pieceP:>(char)colr != colr)  moves:>add(tstpos:>copy());
          break;
        }
      moves:>add(tstpos:>copy());
     }


/* Generate diagonal move for Bishop in Positive X & Negative Y */
tstpos = pos:>copy();
  for ( (tstpos:>(int)x) = pos:>(int)x+1,(tstpos:>(int)y) =
                pos:>(int)y-1;
        ((tstpos:>(int)x) < 8) && ((tstpos:>(int)y) >= 0);
        (tstpos:>(int)x)++, (tstpos:>(int)y)--)
     { pieceP =  (board:>square[tstpos:>(int)x][tstpos:>(int)y]);
      if (pieceP)
        { if (pieceP:>(char)colr != colr)  moves:>add(tstpos:>copy());
          break;
        }
      moves:>add(tstpos:>copy());
     }
return moves;
};



/***********************************************************************/
char Queen::id() {return 'Q';}

object Queen::genmoves()
{ object Rmoves;
  object Bmoves;
  Rmoves = Vmoves();
  Bmoves = DiagonalMoves();
  Rmoves:>merge(Bmoves);
  return Rmoves;
}
/***********************************************************************/
Board board;

main()
{
    free(malloc (1000000));
    system("clear");
    board = Board:>new();
    // Place a piece on the board by declaring it and positioning it as follows:
    object WhiteRook = Rook:>new ('w',4,5);
    object BlackRook = Rook:>new('b',5,5);
    object WhiteBishop = Bishop:>new ('w',2,3);
    object WhiteQueen = Queen:>new ('w',2,5);
    /*
    int i;
    for (i=0; i<100; i++)
    */
    while(1)
    {
        // Each piece will generate its legal moves if so requested:
        WhiteRook:>showmoves();
        WhiteQueen:>showmoves();
        // WhiteBishop:>showmoves();
    }
}

.bp



/********************************************************************
 *        List.e -- Written in ClassC by Paul Kirkaas --
 *
 *        "List.e" creates and defines the abstract data type "list"
 *        for use in the ChessMovs.e program
 **********************************************************************/


#include "List.h"

eqv(a,b) object a,b; /* Compares the eval() strings of the two objects
                      * and returns 1 if they are equivalent; else 0 */
{ return !(strcmp (a:>eval(), b:>(char *) eval())); }

char * ListElement::eval() { return self:>entry:>(char *)eval(); }

List::delete()
{
    head:> delete();
    place:>delete();
}

ListElement::delete()
{
    nextElement:>delete();
    entry:>delete();
}


void Set::add(obj) object obj;
{ self:>rewind();
  object temp;
  while (temp = current())
    {
       if (eqv(temp, obj)) return;
       next();
    }
  ListElement newel = ListElement:>new ();
  newel:>entry = obj;
  newel:> nextElement = self:>head;
  head = newel;
}

void Set::remove(obj) object obj;
{
  object x;
  rewind();
  while (place)
  {  if (eqv (current(),obj)) // The current entry does match obj
        { if (x) // If the previous link is !0;ie, if self is not first element
            { x:>nextElement = place:>nextElement;
              place = x:>nextElement;
            }
          else  // The object was the FIRST element of the list
            { head = place:>nextElement;
              place = head;
            }
         }
      else  // The current entry does not match obj --
         { x =  place;
            next();
          }

    }
}
void Set::merge(set) object set; /* merge two sets */
{ set:>rewind();
  object temp;
  while ( temp = set:>current() )
        { add(temp);
          set:>next();
        }
}

int List::init()
{ head = 0;
  place = 0;
}


object List::current()
{ if ( place) return place:>entry;
  return 0;
}
void List::next()
{ if (place)
    place = place:>nextElement; }

void List::rewind() {place =  head; }

void List::add(obj) object obj;
{ ListElement newel = ListElement:>new ();
  newel:>entry = obj;
  newel:> nextElement = head;
  head = newel;
}

void List::remove(obj) object obj;
{
  object x;
  rewind();
  while (place)
  {  if (  place :> entry == obj) // The current entry does match obj
        { if (x) // If the previous link is !0;ie, if self is not first element
            { x:>nextElement =  place:>nextElement;
                place = x:>nextElement;
            }
          else  // The object was the FIRST element of the list
            {   head =  place:>nextElement;
               place =   head;
            }
         }
      else  // The current entry does not match obj --
         { x =  place;
             next();
          }

    }
}

.bp


/********************************************************************
 *        List.h -- Written in ClassC by Paul Kirkaas --
 *
 *        List.h contains the declarations for the class "list"
 *        which is used in "ChessMovs" and defined in "List.e"
 **********************************************************************/


class ListElement
{ object entry;
  ListElement nextElement;
  delete();
  char * eval ();
};

class List
{ ListElement head;
  ListElement place;
  object current();
  void next();
  void rewind();
  void add();
  void remove();
  int init();
  delete();
};

class Set : List
{ void add();        /* Only adds if equiv object not already in set */
  void remove(); /* Compares by eval string (eqv()); not object # */
  void merge();  /* Combines two sets in one */
};

.bp

.H 2 Backprop.e -- A Neural Network Simulation Program
.na
.nf

/***********************************************************************
 *        Backprop.e -- written in ClassC by Paul Kirkaas
 *        Neural Network simulation program:
 *        Learns exclusive OR through trial and error.  Output reports the
 *        percentage error after each set of one hundred test cases.
 ***********************************************************************/
#include <stdio.h>
#define NO_RUNS 100000
#define NO_PER_SET 100
#define outfile "backprop.rslt"
double exp();
FILE * outFP;
double c = .5;
class Neuron
{
  Neuron to[10];
  Neuron from[10];
  double w[10];
  double q;
  double delta;
  init();
  forward();  /* Applies weights to inputs & generates output */
  back();     /* Examines its given deltas & sends back */
};

double drand48();

int cycle = 0;
main()
{
  Neuron A0 = Neuron:>new();
  Neuron A1 = Neuron:>new();
  Neuron A2 = Neuron:>new();
  Neuron A3 = Neuron:>new();
  Neuron A4 = Neuron:>new();
  Neuron A5 = Neuron:>new();
  connect(A0,A3);
  connect(A0,A4);
  connect(A0,A5);
  connect(A1,A3);
  connect(A1,A4);
  connect(A2,A3);
  connect(A2,A4);
  connect(A3,A5);
  connect(A4,A5);
  srand48(time(0) * getpid());
  int test_pattern;
  for(cycle =0; cycle < NO_RUNS; cycle++)
  {
    test_pattern = input(A0,A1,A2);
    A3:>forward();
    A4:>forward();
    A5:>forward();
    analyze(test_pattern,A5);
    A5:>back();
    A4:>back();
    A3:>back();
  }
}

analyze(pat,nron) int pat; Neuron nron;
{int des =  ((pat & 1) || (pat & 2)) && (!((pat & 1) && (pat & 2))) ;
 double result = nron:>q;
 double error = (des ? (.9 - (double)result) : ((double)result - .1));
 if (error < 0.) error = 0.;
 static double cum_err = 0;
 cum_err += error;
 if (! (cycle % NO_PER_SET))
   {
    printf("%d\t:\t %f %% error\n",cycle,cum_err);
    cum_err = 0;
   }
 nron:>delta += (des - result)* result * (1 - result);
 c = 2*error + .3;
}

connect(obA,obB) Neuron obA,obB;
{ int i = 0;
  for (i=0; i<10; i++)
    { if (! (obA:> to[i]))
        { obA:> to[i] = obB;
          break;
        }
    }

  for (i=0; i<10; i++)
    { if (! (obB:> from[i]))
        { obB:> from[i] = obA;
          break;
        }
    }
}

Neuron::init()
{ int i;
  for (i = 0 ; i < 10; i ++) w[i] = drand48()-.5;
}

double f(net) double net;
{ return 1/(1 + exp( -net));}
Neuron::forward()
// Examines its input, decides on its output
{
 int i;
 double q = 0;
 double net = 0;
 for (i = 0; i < 10; i ++)
     {if (! from[i]) break;
      net += w[i] * from[i]:>q;
     }
 q = f(net);
}

Neuron::back()
// Computes its weight changes and the delta for the previous units
{
 int i;
 double my_delta = delta;
 delta = 0;
 double her_q = 0;
 double my_q = q;
 for (i = 0; i < 10; i ++)
     {  if (! (from[i])) break;
        her_q = from[i] :> q;
        /* First calculate her_delta, becuase it is based on CURRENT weights*/
        /* Use "+=" for the delta, because in the general case, additive */
        from[i]:>delta += my_delta*w[i]*her_q*(1-her_q);
        /* Now calculate weight changes */
        w[i] += c*my_delta*her_q; }
}
input (ob0,ob1,ob2) Neuron ob0,ob1,ob2;
{
  int pat = 4 * drand48();
  ob0:>q = (double)1;
  ob1:>q = (double) (!!(pat&1));
  ob2:>q = (double) (!!(pat&2));
  return pat;
}
