
.ls 2

1) Background -- Began by being interested in Object Oriented programming
   Learn best by creating.

2)    What is object oriented programming?

3)	What does ClassC have/do?

4)	How does it do it?


I became interested in the concept of object oriented programming when I
first arrived at RIT.  I had never heard of the concept before, but Dr
Heliotis mentioned it briefly in class one day, and from that brief
description, it seemed to offer a lot of promise for power and
expressibility.  

As I looked into the concept of Object Oriented Programming more deeply, I
found that there seemed to be as many different definitions of the concept
as there were writers writing about it.

Furthermore, as I began working with some object oriented programming
languages, I found that in the implementation of object oriented languages,
few of them actually offered all of the features
that were described theoretically.  Particularly, I was interested in 
multiple inheritance, which seemed to me to offer a great deal of
potential power.

I worked especially with C++; AT&T's new object oriented version of the C
programming language.  I liked working with this language for three reasons:

1)	It provides many advanced object oriented programming features.

2)	It is based on C, a language which I like a great deal because
	of its simplicity, elegance, and power.

3)	It is very fast.


But again, as I worked with it some more, I found I was dissatisfied
with certain aspects of the language.  It offered no type of multiple
inheritance, and it didn't offer very good dynamic binding.  
These characteristics are related, and I will be
getting back to this later.

At any rate, I realized that the best way to really learn about object
oriented languages and get one that suited my needs was to design and
implement one myself.  This is the motivation behind the development of
ClassC/Elaine.

Before going on, I should discuss a little bit more what is
meant (at least, what I mean) by an object oriented language.

Properly exploring what object oriented programming really means could be a
thesis in itself.  This thesis is about ClassC/Elaine, so I
will just give a sketchy description of object oriented programming here.

There are three qualities which I think are essential to the idea of object
oriented programming.

1)	Data Abstraction

2)	Dynamic Binding

3)	Inheritance


1)	Data Abstraction
	Data abstraction basically means that the user of data doesn't need
to know the details of how that data is implemented.
A familiar example 
typical of most programming languages is arithmetic operations on integers
and real numbers.  Floating point numbers and integers are represented
very differently in a computer, and very different action is taken to
add 5 and 7 versus
adding 5 and 7.2.  Yet the user is shielded from this --  appropriate action
is taken by the computer and the results come out as expected.

This type of data abstraction is hard wired into most object oriented
programming languages.  Object oriented languages allow the user to 
create his/her own new abstract data types.

2)	Dynamic Binding
	Binding is when you associate a symbol to what it really represents.
In general, the earlier you do it, the cheaper it is, but the more you
restrict yourself as well.
Late binding is more expensive, but allows more flexibility.  

Most conventional compiled languages bind a variable to a particular type at
compile time.  That means that variables can only represent data of that
particular type.  Object oriented languages should allow some degree of
dynamic type binding -- at least to the extent that a variable can take on
values of its declared type or types derived from that declared type.

3)	Inheritance
	Inheritance is the ability to define new data types as 
subtypes or derivatives of previously defined types.  The new type 
inherits all the properties and components of the old type, and
adds new components of its own.

-- Another characteristic which is often mentioned in connection with object
oriented programming is \fB Information Hiding\fP.  Information hiding 
means that the user should have no direct access to the raw data
in the object --
the only way to manipulate the object is through the message passing
interface -- that is, through requests to the object.

I don't find this to be a desirable characteristic.  However well
designed an object or abstract data type is, I think there will always be
occasions when you want to do something to an object that the object's
designer had not foreseen.  Ultimately the decision on whether or not
to enforce information hiding depends on the purpose of the language
and the opinion of the implementor.  I chose to allow the user
full access to the object data.



What exactly does all this mean?
How are these abstract properties realized in a programming language?

In a traditional programming system, data exists and is passed to
functions or procedures to be manipulated.  The data itself is passive -- it
doesn't DO anything itself or to itself -- it is acted upon by external
agents (subroutines, etc).

In the object oriented model, the data is grouped together with the
operations that can be performed on it in a single entity called an
OBJECT.  The object, then, can be thought of as a collection of data and
operations on that data.  To perform an operation on the data contained in
an object, a request or a message is sent to the object; and the object
itself is responsible for executing that request.

Different objects may therefore respond individually to the same message or
request.  The issuer of the request does not need to know how that request
is being serviced.

An example:  Suppose you have different shapes -- a cube, a sphere, a
pyramid, etc.  These shapes could be represented by dimensions, etc.

In a traditional programming language, if you have a variable X that
represents one of these shapes and you want to find the volume of 
that shape, you have to have know exactly what shape X represents.
You have to have as many functions for determining volume as you
have types of shapes, and then call the appropriate function for
the particular shape X.

In an object oriented system, there still must be these different ways of
determining the volumes of the different shapes, but the user doesn't need
to keep track of all of them all.  Each object is responsible for knowing
what it is and for having a way to calculate its own volume.  Thus,
the user doesn't need to know exactly what kind of shape X happens to be --
the user just sends a request for volume to X, and X itself interprets the
request, figures out its own volume, and returns the result.

This is an example of data abstraction.

This is the basic essential of object oriented languages, I think.
That is, the ability of different objects to receive the same message
but act on that message individually.  This is what distinguishes 
object oriented languages from more conventional ones.

What is ClassC?

ClassC can be considered a C preprocessor, or a compiler whose object code
is C.  It is can legitimately be considered a compiler because it
is a LALR parser generated by YACC which fully parses the input code,
maintains symbol tables with scoping information, etc.  

Standard C expressions are accepted and usually output unchanged.
	1) Assignment of object variables --
	2) Statement / Declaration mixing --
Expressions which implement features of ClassC are translated into 
C code.  The C compiler is called on the result, and the ClassC runtime
library routines are linked into the executable.  

So what does ClassC do?


  1) ClassC provides true multiple inheritance.

  2) ClassC provides both strict type checking on objects
as well as typeless dynamic binding -- a variable of type
\fI object \fP may be assigned any class instantiation.

  3) ClassC offers automatic garbage collection of memory no longer
referenced by any object variables.

How does ClassC do this?


Objects and classes are implemented in ClassC through standard C structures.
Structures in C are user defined data types which are composed of
a combination of other simpler data types.
A structure in C is defined as in figure 1.

.DS
struct Employee
    {	int	ID;
	char *	name;
	struct date hireDate;
		.
		.
		.
    }	;

.DE

Each component of a structure is named and declared to be of a specific
type.  As in the figure, previously defined structures can be components of
other structures.  Functions cannot be structure components, although
pointers to functions can.

A class is defined in ClassC similarly, except functions are allowed as
members, and a class can inherit components from an arbitrary set of
ancestor classes.

It is the arbitrary number of ancestor classes (multiple inheritance)
that causes difficulty.

To understand why this is a problem, we must first
understand how C accesses structure components.  Structure component
selection in C is done at compile time -- early binding.   For example,
if x is a structure of type Employee, x.name selects the second component of
x, irrespective of what that component really is.  This is okay if x really
is an instance of type Employee, but not so good if the computer thinks
x is of type Employee but is really another type altogether; or if the
component "name" might be located in different positions for different
instances of x.  Let's examine the problems of inheritance.

Inheritance implies that all variables of an ancestor class should
be able to represent members of a derived class.  Going back to our example
of shapes --  there would presumably be a super class "SHAPE", of which
subclasses "cube", "sphere", "pyramid", etc would be derived.  If we have
a system which enforces a reasonable concept of type checking, a variable
myShape would have to be declared to be of super type shape if it we want
it to be able to represent spheres and pyramids and cubes.  This is the
nice thing about inheritance -- it allows type enforcement with flexibility
to represent different types of objects by a single variable of an ancestor
type.  

The assumption of inheritance is that each derived class adds additional
features or components to the ancestor class.  

.DS
figure 2

class Father
    {	int	a;
	int	b;
    }	;


class Son1 : Father
    {	float	q;
    }	;

.DE

This says that the class Son1 is derived from class Father -- that is, Son1
has all the components that class Father does, plus an additional component,
float q.

Take the following example of figure 3

.DS
figure 3

Father myFather;
Son1	mySon;
mySon.b = 5;
myFather = mySon;
print myFather.b;

.DE
This means that if a variable myFather is declared to be of type
Father, we want it to be able to represent objects of class Son1 as well as 
objects of class Father.  We want the expression myFather.b to represent the
same component as the expression mySon.b.

If we are content with simple single inheritance, this is fairly easily
implemented in C by making sure that all the components of the
derived class go after the components of its ancestor class.
Thus the declaration of class Father above could be translated
to figure 4 in C

.DS
figure 4

struct Father
    {	int	a;
	int	b;
    }	;

.DE
And the declaration of class Son1 translated as figure 5

.DS
figure 5
struct Son1
    {	int	a;
	int	b;
	float	q;
    }	;

.DE
In both structures, the component "b" occupies the same relative position.
Thus if the variables myFather and mySon both point to the same object
(a Son1 object) the expressions myFather.b and mySon.b both access the same
component, and all is well.

This is not as easy with multiple inheritance.  Consider the 
new classes Mother and Son2 in figure 6

.DS
figure 6

class Mother
    {
	int	Y;
	int	Z;
    }	;


class Son2 : Father : Mother
    {
	float	r;
    }	;

.DE

How is Son2 translated into a C structure -- figure 7

.DS
figure 7


Is it (A)

struct Son2
    {
	int	a;	/* From Dad */
	int	b;	/* From Dad */
	int	Y;	/* From Mother */
	int	Z;	/* From Mother */
	float	r;
    }	;

or (b)

struct Son2
    {
	int	Y;	/* From Mother */
	int	Z;	/* From Mother */
	int	a;	/* From Dad */
	int	b;	/* From Dad */
	float	r;
    }	;

.DE

Either way you do it, structure dereferencing works for one ancestor and not
the other.


The issue becomes even more complicated if both Father and Mother
have a component of the same name or if they themselves share a common
ancestor.  For example, see figure 8  

.DS
figure 8  


class UniversityMembers
    {
	char *	name;
	int	DateOfBirth;
	void	printStats();
    }	;

class Student : UniversityMembers
    {
	int	StudentID;
   	float	GPA;
	void	printStats();
    }	;

class Employee : UniversityMembers
    {
	int	EmployeeID;
	int	Salary;
	void	printStats();
    }	;

class GradAsst : Student : Employee

.DE


figure 9


.ce
UniversityMembers



.ce
Student			Employee



.ce
GradAsst


Notice that GradAsst inherits the components of UniversityMembers twice--
once from Student and once from Employee; yet we would like only one copy
of those members (name and datOfBirth) to be included in the class
GradAsst.  Thus, for the class GradAsst, the final C structure would
be like the following figure 10:


.DS
figure 10

struct GradAsst
    {
	char *	name;
	int	dateOfBirth;
	int	StudentID;
   	float	GPA;
	int	EmployeeID;
	int	Salary;
    }	;

.DE
This is a very difficult problem to handle with ordinary C
structure dereferencing.  Variables of type Employee and Student would
both expect to find DateOfBirth in the second position; but Student
would expect float GPA in the fourth position; whereas Employee would
expect Salary.  If you had a variable thisEmployee which was supposed to 
be applied to both regular employees and graduate assistants, you
would have a hard time finding out salaries without knowing whether the
current instance of thisEmployee was a grad student or not.  This defeats
much of the benefits of an object oriented system.

This is why C++ currently offers only single
inheritance, although I have heard that Dr Stroustrup, the creator of C++,
is developing a multiple inheritance version.


With ClassC, chose to do runtime symbolic component dereferencing.  That is, 
I developed a method to access components by name rather
than through a standard C structure dereference.  This is more
expensive at run time, but allows greater flexibility and dynamic type
binding.  It is this access method which is the key to the ClassC/Elaine
system.

A class in ClassC is declared as in the figure 11:

.DS
figure 11

class Ship
    {
	int	size;
	int	position;
	int	move();
    }	;


.DE

This results in the creation of a lot of C-code in the output file.
The first thing that is done is to create an array of structures
of type refel; where refel is defined as in the figure 12.

.DS
figure 12


struct _ref_el
    {
	char * name; 
	int offset; 
    };


.DE
The array of refels is as in the next figure 13:

.DS
figure 13

static struct _ref_el _ShipArr [] =
    { 
	{ "size",	(int)&((struct Ship * ) 0) ->size },
	{ "position",	(int)&((struct Ship * ) 0) ->position },
	{ 0 , 0 },
	{ "_name" , (int)"Ship" },
	{ "move",	(int) _Ship_move },
	{  0 , 0 }
    };	


.DE
Note that the array refel is divided into two sections; separated by the
0,0 pair.  Each element in the first section of the refel array pairs
a string to an integer offset within the structure "Ship"; which is
illustrated later.  This is done by casting the integer 0 to a pointer to
structure Ship; dereferencing a component of that structure; taking its
address, and finally casting the result to an integer.  This is a
complicated way of getting the relative offset of that component
into the structure.  For example, the string "position" is paired with
an integer that represents the offset of component "position" in the 
structure "Ship".

The class declaration itself is translated to the
following C structure (figure 14):

.DS

figure 14

struct Ship
    {
	struct _ref_el * _ref_arr;
	int	size;
	int	position;
    }	;



.DE
Note that the member function "move()" is not represented in the 
C structure; note also the component "ref_arr" which is a pointer
to a structure of type _ref_el; or really, the base of the array of
_ShipArr.


The second portion of the array _ShipArr (that following the first
paired {0,0}) pairs each member function name with a pointer to the
location of that function in memory.  Thus the string "move" is
paired to a pointer to the function _Ship_move().

So how does all this fit together?

A dereference in ClassC is written as in the figure 15

.DS

figure 15

Ship	myShip;		/* myShip is declared to be an instance of Ship */ 
myShip = Ship:>new();	/* myShip is initialized */
myShip :> position = 75024;


.DE
where myShip is an instance variable (object) of class Ship.

This is translated into the C as in the next figure 16

.DS
figure 16

* (int *) _deref(myShip, "position") = 75024;

.DE
_deref() is a ClassC library function.  It is this function which does the
runtime symbolic dereferencing of object components as follows:

1)	myShip refers to a structure of type Ship.  The first
	component of this structure is a pointer to the array
	_ShipArr, as in the illustration above.  

2)	_deref() searches this array for the string "position".  
	It finds the string as the first component of the second
	element of the array.

3)	Paired with the string "position" is an integer offset.  
	_deref adds this offset to the starting address of the
	structure referred to by "myShip" and returns it.

4)	The value returned by _deref is cast to an integer pointer,
	and dereferenced.

This is an elaborate way to return a value you would think you could
get just by myShip.position in C; and indeed this system offers no
benefit in the situation just described.  So let's expand on it a little
bit.

Define two new classes, StorageFacility and CargoShip, as in the figure 17

.DS
figure 17

class StorageFacility
    {
	char *	contents;
	int	capacity;
	int	position;
	int	load();
    }	;

class CargoShip : StorageFacility : Ship
    {
	char *	HomePort;
    }	;

.DE

The generated C structure for the class CargoShip would be as the figure 18:

.DS

figure 18

struct CargoShip
    {
	struct _ref_el * _ref_arr;
	char *	contents;
	int	capacity;
	int	position;
	int	size;
	char *	HomePort;
    }	;


.DE
Now, the variable myShip was declared to be an object or instance of
class Ship; which means it should be able to refer to ANY kind of ship,
including a CargoShip, since class CargoShip was derived from class Ship.

But if we set myShip equal to an instance of type CargoShip, it now
refers to a completely different structure than that of simple type Ship.
In particular, if myShip were treated like a structure and dereferenced
by myShip.position, it would return the third component of the structure
to which it points, in this case the capacity of the CargoShip, rather
than its position.

myShip:>position, however, would result in a call to _deref().  _deref()
then looks at the first component of the structure referred to by myShip.
The first component of an object structure is always a pointer to the
array of _ref_el's that pairs the names with the offsets for that class.

It then searches that array, finds the offset for position and returns it
as before.  Notice that the array pointed to by an instance of CargoShip is
different from the array pointed to by an instance of simple type Ship.


Object Table

Lets backup a little now.  What exactly is an object variable?  I said
it \fI refers\fP to an structure in C, but I didn't explain how it
does that.  An object variable is really a C integer which is an index into
a system wide object table.  This object table has a pointer to the 
structure it represents in memory, a reference count field, a dirty flag
(is the entry free), and a class field which identifies the class of which
the object is an instance.  This table is discussed further when I
discuss Garbage Collection.

Member Functions

How are member functions implemented?  Recall our discussion of shapes
earlier.  We want to have several different classes of shapes, each of which
calculates its own volume in its own way, but all of which respond to
the same message "volume".  So if we have three different shapes, we
need to write three different functions to calculate volume; but we
want to be able to access each of those functions with a single message
symbol.  How do we do this?

Consider three classes -- Cube, Sphere, and Pyramid.  They are declared
as in the figure 19:

.DS

figure 19

class Shape
    {
	int volume();
    }	;


class Cube : Shape
    {
	int	lengthOfSide;
	int	volume();
    }	;

class Sphere : Shape
    {
	int	radius;
	int	volume();
    }	;

class Pyramid : Shape
    {
	int	height;
	int	lengthOfBase;
	int	volume();
    }	;



.DE

We now create the volume calculation functions for each of our
three shapes as in the following figure 20.

.DS
figure 20
int Cube::volume()
{ return lengthOfSide * lengthOfSide * lengthOfSide ; }

int Sphere::volume()
{ return 4 * 3.14159 * radius * radius * radius / 3; }

int Pyramid::volume()
{ return height * lengthOfBase * lengthOfBase / 4 ; }


.DE

When compiled by the ClassC parser, these three functions become
translated as follows in the figure 21

.DS

figure 21


int _Cube_volume(self)  int self;
{ return (*(int  *)_deref(self, "lengthOfSide")) *
         (*(int  *)_deref(self, "lengthOfSide")) *
         (*(int  *)_deref(self, "lengthOfSide")) ;
}

int _Sphere_volume(self)	int self;
{
	.
	.
	.
}

etc.


.DE

The refel array for Cube, for example, is as in the following figure 22:

.DS

figure 22

static struct _ref_el _CubeArr [] =
    { 
	{ "lengthOfBase",	(int)&((struct Cube * ) 0) ->lengthOfBase },
	{ 0 , 0 },
	{ "_name" , (int)"Cube" },
	{ "volume",	(int) _Cube_volume },
	{  0 , 0 }
    };	


.DE
So if an object variable myShape of class Shape is declared and assigned
to an instance of class Cube, the message "volume" sent to myShape
via myShape:>volume(); will result in the array _CubeArr above being
searched for the name volume, which is paired with a pointer to the 
function _Cube_volume.  The same message "volume" sent to an instance
of Sphere would search the refel array "_SphereArr" and use the 
string "volume" to find a pointer to the function "_Sphere_volume".

Type Enforced vs. untyped
.br
There are two ways to declare object variables in ClassC.

Declaration by class name --
.br
Classes defined as above can be used to declare object variables.  An object
variable declared by a class name can only be used to represent objects of
that class or descendants of that class.  For example, from the previous
discussions, a variable declared to be of class "Ship" could represent
objects both of type "Ship" and type "CargoShip"; but not objects of
class "StorageFacility".

Declaration by keyword "object" --

Object variables can also be declared by the keyword "object"; which is a
new reserved word / type name in ClassC.  Variables declared by "object"
can be used to represent objects of any class.  This can be useful
when the type of object you want to represent is not known before hand --
for example, in the design of collection classes -- lists or sets.
In C++, for example, you can't just design a class "List";
you have to design it as a class ListOfSomething.  You can't have a
heterogeneous list; that is, a list of objects of different classes.
All the objects in a C++ list have to be of a single class or derived from
that single base class.  

This is not such a terrible disadvantage.  Frequently the fact that you can
design a list that accepts derivatives of the the base class as well as
objects of the base class itself allows you to have a list that offers a
sort of heterogenaity that is sufficient for many applications.

Sometimes, however, this may not be enough.  We may wish to manipulate
objects with no \fIa priori\fP knowledge about their class at all.
In this case, we can use variables declared to be of type "object".
We can design lists of objects, sets of objects; etc.

There is a difficulty.  In other object oriented systems that provide this
ability; Smalltalk, for example, EVERY entity in the environment is an
object.  ClassC adds objects and classes on top of the existing structure
of C; so objects and integers and pointers, etc must all co-exist.
This can be a source of difficulty when dereferencing a variable of type
"object".
If we declare the variable "genericObject" as type "object",
it can represent instances of any class at all.
Therefore, there is no way for the compiler to know the type of
genericObject's components.  The default, therefore, is that objects
declared to be of type "object" return type object components.  Since this
is not always the case, the programmer has two options.

When the programmer dereferences an object, he/she presumably know what type
of component to expect, even if the compiler does not.  Therefore, the
programmer can instruct the compiler to expect a return value of any
specified type by an expression similar to a standard C cast.  For example,
if the expected component type of an object variable "Board:>field"
is an eight by eight array of integers, the appropriate dereference is as in
the figure 23:

.DS

figure 23

board:> (int [8][8]) field

.DE
which gets converted by the compiler to the following figure 24 in C:

.DS

figure 24

* (int * [8][8] ) _deref(board, "field")


.DE

This solution is often the simplest, but is bad programming style because
the programmer is responsible for knowing the type of the object and its
expected return types.  This is exactly the sort of thing that object
oriented programming was intended to replace.  So a more formal way of
manipulating untyped object variables is to query them.  The expression
in the figure 25

.DS

figure 25

objectVariable:>_name


.DE
will return a string containing the name of the class of which that object
is an instance.  The user can then use that knowledge to manipulate the
object appropriately.

In any case, the use of untyped object variables is awkward.  In some cases
it may be necessary or desirable to use this feature, but in general it will
be far more practical to use the first described named class object variables.


Garbage Collection -- What is it?

First, I must reemphasize the difference between objects and object
variables.  Object variables, as I indicated before, are really integer
indices into an object table.  An object variable only ever takes up
a single word of memory.  The object to which an object variable
refers, however, is allocated from the heap and may be arbitrarily large.
Usually, one object variable refers to one object in heap, and that's that.
It is possible, however, for several object variables to point to the same
object, or for NO object variables to point to a particular object.
It is this last case which can cause trouble, because when we have an object
with no object variables pointing to it, we have no way of referencing it or
doing anything with it again.  It just sits there in memory forever, taking
up space and doing nothing.  Garbage collection is the process of going
through all the object on the heap, finding those "floating" objects that
no longer have any references to them, deleting them, and releasing
their memory back to the heap to be used again by new objects.


This is a nice feature, but it is difficult to implement, involves a lot of
run time overhead, and frequently un-necessary.  Many object oriented
languages provide Garbage Collection (Smalltalk and Owl); many do not
(C++ and Objective C).  
But since no other C based object oriented system I know
of does provide automatic garbage collection, I decided
to try and develop such a system and offer it as an option that can be
selected at compile time.  

I implement the ClassC garbage collection mechanism through reference
counting.  In the object table, there is a reference count field.
Every time an object is assigned to an object variable, the reference
count of the object the object variable \fI used \fP to point to is
decremented, and the reference count of the object which has just been
reassigned is incremented.

Assume a,b & c are object variables.  Without the garbage collection option
selected, the expression a=b in classC is translated directly as a=b in the
C output file.  With garbage collection, however, the expression a=b is
translated as in the figure 26:

.DS

figure 26

* _decref (&a ) = _incref (b)


.DE
This is to allow the return value of the function _decref to produce an lvalue.

Cascaded assignment like a=b=c is translated as in the figure 27

.DS

figure 27

*_decref( &a ) = _incref (* _decref (&b ) = _incref (c) )


.DE
Whenever a new object is created, a counter is incremented.  At every
hundred ticks of the counter (that is, after each hundred new objects
are created) the object table is garbage collected -- that is, it is
traversed by the Elaine library function _collect() and all objects with
zero reference count are free back to the heap.


That is the essence of the ClassC/Elaine system.
