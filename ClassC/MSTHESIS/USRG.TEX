\section{Functional Specification --- ClassC/Elaine Programming Guide}
\subsection{Introduction to ClassC/Elaine}
ClassC is an object oriented programming language based on
{\em C},
and it is assumed the user has a both a solid knowledge of the C
language and a conceptual understanding of object oriented programming.

Like C++ and ObjectiveC, ClassC is really a preprocessor, converting ClassC
source code into C, then into assembly language, and finally into machine
code.  ClassC is a superset of the C language, and all C constructs except
{\em enum} are accepted and semantically unchanged ({\em enum}
was not implemented because it differs from other aggregate types.  It would
require a special symbol table, and is not widely used in C.  It would
be a tedious but not difficult feature to add).

{\flushleft ClassC/Elaine has the following features:}

{\flushleft \large \em ClassC's object oriented extensions to C:}\\
\begin{description}
\item[Class Definition:]
A class is defined like a C structure, except that it may inherit
components from an arbitrary list of ancestor classes, and may include
functions as components.
\item[Objects:]
An instance of a class as defined above is an
{\em object}.
Objects can be declared in one of two ways.  A class definition is like a
typedef in C.  This means that a variable can be declared by using
a class name as a type declaration.  Variables so declared can only be
assigned objects of that same type or ancestor types.
In addition, a new data type
{\em object}
is introduced, which can be used to declare variables which represent
{\em any}
class type.
\item[Optional Inline Object Dereferencing:]
Invoking the {\bf ce} compile command with the option {\bf -i} compiles
the ClassC code with inline rather than funtion-call object dereferencing.
The mechanism is explained in the Implementation section, but the
effect of inline dereferencing is software that runs faster when it runs at
all, but uses such subtle tricks of the underlying C compiler
that it is not guaranteed to be reliable.  Use of the standard function
dereferencing mechanism is generally recommended.
\item[Garbage Collection Option:]
Invoking the {\bf ce} compile command with the option {\bf -G} results in the
inclusion of the ClassC garbage collection mechanism.
\end{description}

{\flushleft \large \em ClassC's non-object oriented extensions to C:}
\begin{description}
\item[Inline Comments:]
Following C++, any text between ``//'' and the newline is ignored.
\item[Statement/Declaration Mixing:]
Statements and variable declarations can be intermixed in the body of a block,
as long as each variable is declared before its use.
\end{description}

The distinction between classes, objects, and object variables
in ClassC should be understood.
A class type definition is used to define the valid data
components and operations (methods, messages) for objects of that class.
An object is an instance of a particular class,
created dynamically from the heap at runtime by explicit calls to
{\em $<$className$>$ :$>$ \tt new()}.  An object
{\em variable}
is implemented as an integer index into an object table.  Since the type
(class) of each object is kept in its object table entry, it need not be a
characteristic of the object variable itself.  This allows dynamic binding of
object variables to objects.

In other C based object
oriented languages, the type of the object variable must be declared to be
of a specific class, and can be used represent only objects of that
particular class, or classes derived from that class.  In ClassC, type
checking is performed only on those objects declared to be of a particular
class.  The declaration
{\em object}
can be used as a catch-all to hold any system object type with no
compatibility checking.  This is useful for collection classes such as
sets which should be able to contain all object types.

\subsection{Overview}
This section introduces the flavor of programming in ClassC.  More detailed
explanations can be found in the following sections.

Objects in ClassC are much like structures which can have function components.
The ``:>'' operator is used to dereference a component of an object.  If the
component is a data/variable type component, it can be used as an l-value ---
that is, it can appear on the left side of an assignment expression.  There
is no concept of ``private'' components in ClassC.

A function component in ClassC is like a message in Smalltalk.
Messaging could have been implemented with a Smalltalk-like syntax, but
one goal of ClassC was to be as C-like as possible while providing object
oriented extensions.  Hence, messages arguments are passed like regular
function arguments in C.  Thus the expression:
\begin{quote} \tt
thisObj :$>$ message(arg1,arg2);
\end{quote}
applies the member function {\em message()} to the object ``thisObj''
with the arguments {\em arg1} and {\em arg2}.

In ClassC, classes are defined by the inheritance of components from
parent classes and an additional declaration section for components unique
to the class being defined.

A class name may be used to declare an object variable, which then represents
a pointer to an object of that class type.  However, the object itself is not
created until an explicit initialization call is made.

For example, the declaration
\begin{quote}
MyClass objVariable;
\end{quote}
declares ``objVariable'' to be a pointer to an object of type ``MyClass''; but
it is initialized to the NULL pointer (0).  To create a new object of type
``MyClass'', send the message ``new()'' to ``MyClass'' ---
\begin{quote}
MyClass:$>$new()
\end{quote}

It is possible to  declare an object variable and initialize the object
in one step:
\begin{quote} \tt
MyClass objVariable = MyClass:$>$new();
\end{quote}

Class names are also used when defining member functions.
Since several
classes may have member functions of the same name, the class name is
prepended to the member function name to uniquely identify it.

\subsection{Defining classes}
A class in ClassC is defined much like a structure in C.  A class definition
is indicated by the keyword {\bf class}, followed by the name of the class to
be defined, an optional inheritance list of previously defined classes, and
the body of the class definition.  The definition body is just a sequence of
declarations in a standard C format.  This declaration set may include
function and object declarations.

The form of a class definition is thus:
\begin{verbatim}

class Class_name  : SuperClass_1  : SuperClass_2 : ...
	{	<type> <declarator>;
	 	<type> <declarator>;
	 	<type> <declarator>;
	 	<type> <declarator>;
			.
			.
			.
	};

\end{verbatim}


Three things to bear in mind about classes --
\begin{description}
\item[Class Definitions ---]
No other classes, structures or typedefs may be
{\em defined}
within a class definition, although previously defined typedefs
and structures may be used
within a definition, and other classes which are defined elsewhere
within the file (earlier or later) may be used to declare member components.
\item[Class Names ---]
The ``class'' keyword is used
only on definition.  Thereafter the defined class name is used like a
new type name.  This means that unlike structures,
you cannot have a variable with the same name as a class.
\item[No Unnamed Classes ---]
Unlike
{\em struct}
in C, all classes must be given a name; furthermore, it is impossible to
declare an object instance of a class
in the same statement as the class definition.
\end{description}

{\flushleft \large \em Class Inheritance:}\\
Classes may inherit components (data and methods) from an arbitrary number
of antecedents.  This implies a potential conflict when the same component
name is used in more than one ancestor.  This conflict is resolved as
follows:
\begin{itemize}
\item If the conflict involves two components of the same name but of different
declared type in two or more different ancestors, this is declared
to be a semantic error,
compilation is aborted and an appropriate error message is reported.
\item If the conflict involves components of the same name and the same
declared type in each ancestor, then the component from the parent
earliest in the list is defaulted.

This default inheritance of the component is irrelevant for
data components, but critical for function components.  That is, if the
same function name and type is declared in several ancestors, the first
definition encountered when traversing the list of ancestors is the
function definition used for the new class.
\item
If the conflict involves a component explicitly declared in the
body of the current class definition and an ancestor of that new class, the
new definition is used, which supersedes all ancestor definitions.
This is the case even if the type of the new component declaration
conflicts with the type declaration of a component of the same name in an
ancestor class.
\begin{quote}
Note that if the user wants to supersede a function
definition from a parent class, that function must be declared in the
{\em body}
of the new class as well as being redefined as described below.
\end{quote}
\end{itemize}

\subsection{Defining class functions}
1) {\large \em Declaration}\\
Different classes may have member functions with the same name.
Similarly, a derived class may redefine a function which was defined
previously in an ancestor class.  It is therefore necessary to have a means
to distinguish which of the several possible member functions is being defined.
This is done with the class name and the function name joined with the
scoping operator ``::''.  So to define the member function ``print()'' of class
``Bishop''  which takes a char * argument ``str'' and returns an integer ---

\begin{verbatim}
	int Bishop::print( str )
	char * str;
	{
		(Body of definition)
	}

\end{verbatim}


{\flushleft 2) {\large \em Body}}\\
The body of a member function definition is like the body of an ordinary
function body in ClassC, except in the treatment of the
components of the object to which the message/function is being sent.

Specifically, the body of a member function definition can refer to other
elements of the receiving object without explicitly dereferencing them.
The appropriate dereferences
are implicit in the function itself.  For example:
\begin{verbatim}

	class Piece
		{ char *	name;
		  object	position;
		  int		checkPosition();
		  int		show();
		  void		move();
		  void		report();
				/* Send message to coordinator */
		};

\end{verbatim}
is a declaration for the class piece.  The member function ``move()'' of class
Piece might be defined as follows:

\begin{verbatim}

	void Piece::move( newPosition, board )
	object newPosition;
	object board;
	{    if (checkPosition (newPosition ) )
	     {
		board:>atput(newPosition, self);
	        position = newPosition;
	      }
	     else report ("Illegal Move");
	}


\end{verbatim}

Note the use of the identifier
{\em self}
in the definition above.  Self is an object variable which is automatically
generated by the parser as an argument to the member function.
It represents the target object to which the function/message is being
sent.  The variable name {\em position} used in the definition of
{\tt Piece::move()} above is known to be a component of class Piece, and is
therefore assumed to refer to the target object.   The statement
\begin{quote}
position = newPosition;
\end{quote}
could have been equivalently rewritten
\begin{quote}
self :$>$ position = newPosition;
\end{quote}
In general, the {\bf self} variable is used when the member function manipulates
the
{\em entire}
object rather than the components of that object.
In this case, for example, it is used to register the
instance of the Piece class to its new position with the object {\tt board}.

\subsection{Declaring and Initializing Objects}
Objects can be declared in one of two ways:
\begin{enumerate}
\item By the type name {\em object}, or
\item By the name of a class defined elsewhere in the file.
\end{enumerate}
When an object variable is declared by a class name, all operations
subsequently performed on that variable are checked for conformance
at compile time.
That is, operations are checked to ensure that they are compatible with
the object's declared class or ancestor classes, and the returned
result of the message is cast to the appropriate type.

Variables that are just declared to be of type
{\tt object}
are not subject to any class conformance checking until runtime.  Thus, any
object operation may be applied to variables declared to be of type
{\tt object}.
This gives the user more flexibility in the manipulation of the object,
but at the risk of runtime errors if the user has used the object
incorrectly.

It is suggested that use of the type name
{\tt object}
be restricted to such applications as collection classes, where it is desirable to
allow objects of all types to be stored.

These object type names are syntactically equivalent to
standard C scalar type names like ``int'', ``char'', ``float'', etc.
This means that they
can be used not only to declare variables of type
{\tt object},
but arrays of, pointers to, and functions returning
{\tt object}
as well.  So some possible object declarations:
\begin{verbatim}
	object	x;
	MyClass	y[4];
	object	(*z)[3][7];
	Bishop	func();

\end{verbatim}

A new object variable is essentially an empty slot into which an object
may be placed.  Objects themselves are created dynamically from the heap
at runtime by a special call or message called {\bf new()}.  The message
``new()''
is sent to a class name to create a new instance of that class.  This new
instance may then be assigned to an object variable (be put in the slot).
\begin{quote}
Note: Classes in ClassC are not themselves objects, but are syntactically
treated as such with the ``new()'' message.
\end{quote}

So, for example, to make a new object of class Bishop you write
{\tt Bishop:$>$new()} where the symbol ``:$>$'' is the message passing
or dereferencing operator of ClassC.

An object variable may be initialized on its declaration line like any other
C variable.  So, to assign the new instance of Bishop to the object variable
{\em mypiece}
we could write either:
\begin{verbatim}
	Bishop mypiece;
	mypiece = Bishop:>new();
\end{verbatim}

or

\begin{verbatim}
	Bishop mypiece = Bishop :>new();
\end{verbatim}

It is also possible to create a special {\bf init()} function for a class
which is automatically called when a new object of that class is created.
The ``init()'' function is declared and defined like any other class member
function except that it must not be declared to be of any type or return
any value.  An ``init()'' function may be defined to take arguments,
in which case the call to
\begin{quote} \tt
ClassName:$>$new(arg1,arg2,...)
\end{quote}
is made with the appropriate arguments.

New objects created from classes that have no "init()" function associated
with them are created with uninitialized memory.


\subsection{Deleting Objects}
Primitive garbage collection is performed automatically when the ClassC {\bf -G}
compiler option is used.  The user can also explicitly delete unwanted
objects and release their memory space to the heap.

An object is deleted by sending it the message {\bf delete()} ---
\begin{quote} \tt
oldObj:>delete();
\end{quote}

The function ``delete()'' does not have to be declared as a component of
an object to be used to delete an object.  It can be used to delete
any object, and just releases the memory specifically allocated to
that object by the call to ``new()''.

It is possible, however, that some objects may have acquired additional
memory resources.  For instance, an object may contain a pointer to a
dynamically allocated string.  If the object is deleted without freeing
the string, the string will remain, uncollected, floating in memory.

It is thus possible to explicitly define a member function ``delete()''
as part of a class definition.  This member ``delete()'' function is
defined like any class member function, and the user is responsible for
cleaning up any additional memory allocated to the object here.
When the message ``delete()'' is sent to an object (possibly with
arguments) the user defined member ``delete()'' function is called to
clean up user allocated memory, then the system ``\_delete()'' function is
called to delete the object itself.


\subsection{Using Objects}
There are three different ways in which object variables can be used.
\begin{itemize}
\item
They can be used with the dereferencing (component selection) operator
``:$>$'' ---\\
e.g., {\tt mypiece:$>$position = newPosition;} or
{\tt mypiece:$>$set(newPosition);}
\item
They can be used in assignment statements with other objects or functions
returning objects;\\
e.g., {\tt obj1 = obj2;}.
\item
They can be passed as arguments to functions; e.g.
{\tt newobj = transform(oldobj);}.
\end{itemize}


{\large \flushleft \em Dereferencing Objects}

{\flushleft Member Data:}\\
The operator ``:$>$'' is used to dereference or select components of an
object.  If this component is a data component, the expression
\begin{quote} \tt
$<$object\_name$>$ :$>$ $<$component\_name$>$
\end{quote}
is an l-value;  that is, the expression
can be used on either side of the ``='' (assignment) operator.

Components of a class can be of any legal C type, or a class C object.
If the object variable has been declared with a class name, component type
interpretation is done automatically.  This means that if object
{\tt obj1} has been declared to be of class SampClass and component
{\tt mycomp} is an array of floats, the assignment
\begin{quote} \tt
obj1:$>$mycomp [6] = 2.718;
\end{quote}
has the expected result.
But as discussed above, object variables declared as type
{\em object} (rather than as instances of previously defined class types)
are dynamically bound and
their type cannot be determined at compile time.  Thus, there is no
way to determine the type of component {\tt mycomp} at compile time.

{\em Example:}\\
Consider two classes, ClassA and ClassB.  Each class has a component named
{\tt size}.  The declaration of ``size' for class ClassB is {\tt int size;}
for ClassA it is {\tt float size;}.  If the object variable {\tt objVar} is
declared to be of type
{\em object},
it is not bound to any class type at compile time and the compiler
has no way to know the type of the expression
{\tt objVar:>size}.  This is resolved in the following way:

\begin{itemize}
\item
The default return type of the ``:$>$'' operator is type {\tt object}.  Thus, if
an object is expected, no special treatment is required.
\item
If an abstract declaration appears between the component selector and
the component, the type of the expression is cast to the type of the
abstract declaration.  The abstract declaration is basically a C cast except
that the conversion is {\em exact}.
That is, if the component is an array of 6 floats, the abstract declaration
for that component MUST be ( float [6] ).  In C, the appropriate
cast would be ( float * ); the cast ( float [6] ) is illegal in standard C.
Similarly, if the desired component is a structure, it must be cast as the
structure rather than a pointer to structure.  So if the expected return
type of {\tt objVar:>size;} is float, it must be expressed as
{\tt objVar:>(float)size}.
\end{itemize}

This interface is awkward, but
is provided as an additional feature for the user.

{\em When an object variable is declared as an instance of a particular class
rather than with the keyword
{\bf object},
these conversions are performed automatically and need not be of concern to
the user.}

\subsection{Use of Member Functions}
Member functions are accessed through the same ``:$>$'' selection operator by
which member data are selected.  A member function can access other member
functions within its body; in particular, member functions of a child
can access and call functions inherited by that child from its ancestors.
Similarly, an inherited function that references function components that
have been redefined in the derived class will call the redefined
{\em derived}
class function rather than the function for which it was originally
defined.  As with data components, function component return values
are automatically cast to their declared types for object variables
declared by a class name, and can be manually cast as described above
for object variables declared by the keyword {\bf object}.
\subsection{Object Assignment}
Objects can be assigned to object variables through the standard C
assignment operator ``=''.  This means a
reference between the object variable and the object itself is
established.  No new objects are created by this mechanism.  For example,
if ``a'' and ``b'' are both object variables, then the assignment ``a = b;''
does NOT duplicate the object pointed to by b.  Rather, both ``a'' and ``b'' now
reference the same object.  Thus, any changes made to object ``b'' will be
made to object ``a'' as well.  In order to duplicate an object, the user must
allocate new memory and initialize it appropriately,
usually through a member function.  For example, to create a duplicate
of object {\tt objGen}, one might write
\begin{quote} \tt
AClass dupObj = objGen:>copy();
\end{quote}
where {\tt copy()} has been appropriately defined by the user to create,
initialize, and return a new object.

\subsection{Using the -G Garbage Collection Option}
Garbage collection is useful for large programs that use large amounts of
memory, or programs with repeating loops that reassign object variables
without explicitly freeing old values.  Garbage collection is a feature
which allows the programmer to be less rigorous when writing his/her
program, but is costly in terms of computer resources and should be used
judiciously.

Note that when the garbage collection option is invoked, objects
are statically scoped, and hence are not stacked in recursive function
calls.  Note also that modules compiled with the garbage collection
option must not be combined with other modules compiled without it.

Garbage collection takes place only after a certain number of new objects
have been allocated, and thereafter only at specified intervals.
Both the initial threshold and the frequency can be modified from their
default values by the user.

The external integer ``CEGthreshold'' is the number of new objects that
will be allocated before the garbage collection mechanism is called for the
first time.  On systems with large memory, it often makes sense to allow
this number to be quite large before trying any collection.  The value
can be changed in any function by declaring the variable
\begin{quote} \tt
extern int CEGthreshold;
\end{quote}
and setting it to the desired value.

The frequency of the collection is modifiable as well.
{\em Frequency}
of collection,
{\bf CEGfrequency},
is defined as how many calls to
new()
are made between collection attempts.  It is wasteful to
collect the entire object table every time the user wants to allocate a new
object.  After the threshold size
CEGthreshold
has been reached, collection takes place
after each {\em CEGfrequency} calls to new(), where CEGfrequency is declared as
\begin{quote} \tt
extern int CEGfrequency;
\end{quote}
That is, the integer
{\em CEGfrequency}
represents how many calls to
{\em new()}
are made between each attempt at collection of the object table.

Again, it is only necessary to declare the variable CEGfrequency
if it is desired to
modify its default value.  The value may be changed by assigning a
new integer to it.

The default value of CEGthreshold is 1000; default for CEGfrequency is 100.

\subsection{Using ClassC/Elaine}
The class definitions should in be put at the beginning of the source
file, and should occur before any function definitions.

If a large multi-file project is being developed, all class definitions
should be made in header files which are
{\tt \#include}'ed
in each file that manipulates objects of those classes.  The member
functions of each class should be defined only once, and
can be defined in any file and linked in at
load time.

The
{\bf ce}
command invokes the ClassC compiler and processes command line options.
The only options special to
ce
is the garbage collection option {\bf -G} and the dereferencing options
{\bf -d} and {\bf -i} for function dereferencing and inline dereferencing,
respectively.

Other command line options implement standard cc options as described below:


{\em ce command line options}
\begin{description}
\item[-O]
Invoke the C optimizer on the output.
\item[-o {\em name}]
Call the executable file {\em name}.
\item[-g]
Compile for debugging.
\item[-p]
Compile for profiling.
\item[-G]
Create with garbage collection.
\item[-e]
Create .c intermediate file; do not compile into object file.
\item[-c]
Create object file; do not link into executable.
\item[-l{\em libname}]
Search the library ``lib{\em libname}.a'' for unresolved references
when linking.
\item[-i] inline dereferencing
\item[-d] deref --- function dereferencing
\end{description}

Use the ``-c'' option and the
Unix ``ar'' library archive command to build ClassC libraries.

\subsection{Keywords and special symbols}
\begin{verbatim}

:>		--	Class dereference operator
::		--	Class scoping operator
:		--	Inheritance operator
//		--	ClassC inline comment operator
class		--	Class definition keyword
object		--	Object type name
self		--	Target object in member function definition
_deref		--	ClassC library dereference function
_new		--	ClassC library object creation function
__new		--	ClassC library object creation function
_decRef		--	ClassC library reference count decrement
_incRef		--	ClassC library reference count increment
_delete		--	ClassC library delete function

\end{verbatim}

\subsection{Examples of ClassC programs}
See appendices.
%\clearpage
