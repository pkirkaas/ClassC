

Thesis Proposal for:

.ce
.I ClassC/Elaine

.ce
A Multiple Inheritance C-based Object Oriented Language


Paul Kirkaas

.ce
.I "Table Of Contents"
.bp
.H 1 "Introduction"

ClassC is a C-based object oriented programming language.  The language
adds object oriented capabilities to the C programming language.
Specifically, ClassC adds the new data type
.I object
, and the new aggregate declaration,
.I class.
Class definitions are like C
structures except they may contain function declarations, and they may
inherit components from multiple ancestors.  Objects are instances of
a particular class.

.I Elaine
is the programming environment for ClassC.  Initially it
consists of the runtime library functions necessary to implement
ClassC; enhancements will be added over time.

There are other C-based object oriented languages, such as C++ and
Objective-C, but these languages are more traditional than ClassC.  ClassC
differs from these other languages in two points.

  1) ClassC provides multiple inheritance, which is very commonly
discussed in the description of object oriented languages, but very
rarely actually implemented.

  2) ClassC objects are typeless and dynamically bound -- that is, any
variable of type
.I object
may be assigned any class instantiation.  This is typical of interpreted
languages like lisp and smalltalk, but rare in compiled languages like
.I C
and
.I Pascal,
etc.

.H 1 "Motivation"

Object oriented programming is a way to abstract data and
control.  It allows the programmer to create models that are internally
complex and sophisticated, yet which present a simple and uniform interface
to the user.  There are two broad arguments for Object Oriented
programming.

The first is that Object Oriented programming
improves productivity and reliability by allowing the reuse of code and the
standardization of common data types and operations.

The second argument is that by allowing greater abstraction, object
oriented languages are continuing the trend in programming languages
towards a more human like and less machine like means of expressing
concepts.  This implies that object oriented programming can facilitate
not only bigger programs, but cleverer ones as well.  This is my primary
interest.

The goal of ClassC/Elaine, then, is to provide a usable C based object
oriented language that offers an alternative to previous implementations.
Most of the code executed in any program is straight forward and does not
benefit from an object oriented approach ( e.g, for-loops, arithmetic
operations, etc).  This code should be implemented as efficiently as
possible for rapid execution.  This is the motivation for using C as the
base language.  Object type data manipulations should be used for operations
which are conceptually intricate, where use of object oriented techniques
could give both the programmer and the program more clarity and
expressiveness.  This is the motivation for enhancing C with object oriented
features; moreover, this is the reason for giving ClassC a higher level of
abstraction that C++ or ObjectiveC through multiple inheritance
and untyped objects.

In order to be genuinely useful, ClassC should be compatible with current
C language facilities.  It should be compatible with all existing standard
.I "#include"
files, as well as all library packages.  It should allow separate
compilation and linking of ClassC modules, and should allow the user
to create his/her own ClassC libraries.

.H 1 "Background"

The foundations of Object Oriented programming were laid by the
programming language Simula, which was developed in Norway in the early
1960's[XXX].  It introduced the concept of user defined classes, and was
developed as a simulation language for modeling complex systems.
The power of this concept was not widely recognized until Alan Kay and the
research group at Xerox PARC developed the Smalltalk interactive
environment[GOL83].  It evolved from Dr. Kay's concept of a computer called the
Dynabook, which would be very user friendly and suitable as a learning
aid for children.  Smalltalk remains to this day the prototypical example of
an object oriented language.  It is a rich and complex environment that
shows the power of the object oriented concept.

Yet this richness and complexity has a price.  Smalltalk is big, slow,
expensive, and available on very few machines.  It's primary drawback from a
systems development perspective is its slowness.  Smalltalk runs on a
virtual machine / interpreter, which is inappropriate for software
systems.  This is unfortunate, because software systems are classic
candidates for the productivity benefits of object oriented programming.
Concepts like queues, stacks, hash tables, etc. are ideal abstraction
types.  But an operating system written in Smalltalk running on conventional
hardware would be very slow.

Enter C++ [STR86].
C++ was developed at Bell Labs, home of Unix, C, Kernighan and
Ritchie.  It was developed by Bjarne Stroustrup with the intent of giving as
much object oriented structure to C as possible, while making no sacrifice of
the efficiency of pure C code.  It is therefore perhaps the best system
development language available today.

Objective-C is a language developed by Dr. Brad Cox at Productivity
Products International and seems much like
C++ with a different syntactic structure.
The two languages offer similar features [COX86].


.H 1 "Project Description"

My concept for ClassC differs from those of other C-based object oriented
languages.  I am interested in developing a language that is a
learning tool for data abstraction and object oriented programming
concepts.

Consequently, I emphasize the object oriented flavor at the sacrifice of
some efficiency.  This philosophy is reflected in the decision to provide
multiple inheritance and dynamic binding of objects.  There is some runtime
cost to provide these features, mainly function call overhead.

ClassC will be a superset of C; that is, all C operations are supported by
ClassC, and pre-existing library and object files as well as standard
header files will all be compatible with ClassC.

The language will be developed in a Unix operating system environment, using
the Unix language development tools.  The parser will accept standard C code
and pass it to the output file unchanged.  It will need to maintain symbol
tables and scoping information and build a parse tree, however.

In addition to standard C, the parser accepts and processes the following:
.AL
.LI
The aggregate definition
.I class-
.br
A class declaration is much like a standard C structure declaration.
A class is composed of
.I Members;
of which there are two types:
.AL a
.LI
.I "Variable"
 members-
.br
which are passive data components of the class instances.
.LI
.I "Function"
members-
.br
which are functions that act on the private member variables of class instances.

.LE
In addition,
a class definition may contain a list of parent classes from which
components (variable and function) are inherited.
.LI
The class member function definition-
.br
Different classes may have member functions with the same name; e.g.,
.I print.
To distinguish them, class member functions are defined using the class name
and a scoping operator.
.LI
Objects-
.br
All objects are dynamically allocated at runtime by explicit calls to
a
.I 'new()'
function.
A variable of type
.I object
is really an index into an object table.
An object may be declared in one of two ways:
.AL
.LI
With a class name-
.br
Class names are used to declare object variables.  Object variables declared
this way can only represent objects of that particular
class type or its descendants.
.LI
With the type specifier
.I object-
.br
A variable declared with the type specifier
.I object
can be used to represent any type of object instance at all.
.LE
.LI
The object member dereference-
.br
The member dereferencing operator used in ClassC is ":>".
C++ uses the ordinary C structure dereferencing operators for object
dereferencing, but the dereferencing mechanism in ClassC is implemented
differently and suggests the use of a separate symbol.
.LI
Variable declarations-
.br
Variable declarations can be made anywhere in a block.
This is intended to more closely associate the variable declaration
with its use.  Every variable must still
be declared
.I before
its first use.
.LE

.H 1 "Functional Overview"

.I Classes:
.br
A class is declared as follows:
.DS
class <Class_name>  : <SuperClass_1>  : <SuperClass_2> : ...
	{	<type> <declarator>;
	 	<type> <declarator>;
	 	<type> <declarator>;
	 	<type> <declarator>;
			.
			.
			.
	};

.DE
where
.I SuperClass es
are optional and separated by the inheritance operator ":"; and where
.I declarator
is any type of legal ClassC declarator, including a function
or a class object.

A special type of
function that may be declared as a class member is
.I init().
.I init()
is that class's initializer function.  The init() function is defined by
the user and is automatically invoked when an
object of that class is created.  Otherwise, if init() is not declared,
uninitialized memory space will be allocated.

.I Objects:
.br
Objects are created dynamically from the heap; there are no static objects.
Each object is an instance of a specific class.
A member of an object (the member can be either variable or function)
is dereferenced by the object dereferencing operator ":>".  The general
form of the object dereference is:
.br
"<objectExpression> :> <memberName>;";
.br
where "objectExpression" is usually just a variable name.  In general,
however, an object expression can also be the result of a pointer
dereference, a return value from a function, etc.
The type of the entity returned by the object dereferencing operator is
determined by the type of the object expression.  If the class to which
the object
expression belongs can be determined at compile time, the
type of the returned dereferenced element is determined by its type as
declared in the class definition.

If, however, the object expression evaluates to an instance of the
generic class
.I object,
its true class cannot be determined at compile time.  In this
case the type of the result of the dereference is defaults to
.I object
as well.  The programmer can cast the result to be of
any type and override the default.

.I "Member Functions:"
.br
Member functions are defined with the class name, scoping operator "::", and the
function name -- e.g, "Son::print()".   Arguments are declared just like in
an ordinary C function definition.  An object member function can reference
the other
member variables or functions
of that same object with no special dereferencing or casting
required.  A reference to "self" is passed automatically on a call to a
member function, and may be used explicitly by the programmer
in an object's member function definition to
reference the object itself.


.br

Examples:

Declare class
.I Son
to be a descendant of classes
.I Papa
and
.I Mama.

.DS
	class Son : Papa : Mama
		{ int a;
		  char * b;
		  init();	/* Optional initialization function */
		  void print();
		};
.DE

Define member function init() of class Son:
.DS

	Son::init(x,y)	int x; char * y;
	{ b = (char *) malloc(strlen(y) + 1);
	  strcpy (b,y);
	  a = x;
	}
.DE

Declare an object & initialize to be of class Son:

.DS
	Son xmpl = Son:>new(5,"Hello");
.DE

Send the message "print" to xmpl:

	xmpl:>print();

Reset member element "b" of xmpl:

	xmpl:>b = "Goodbye"
.bp

.H 1 "System Description"

The ClassC/Elaine compiler will be developed under Unix using the Unix
lexical analyzer
.I Lex
and the Unix parser generator
.I Yacc,
and will use the standard Unix/C libraries.  The final system will consist
of three files-
.AL
.LI
The parser
.Iefront-
.br
The ClassC/Elaine front end parser, written in C.  The parser will translate
ClassC/Elaine .e source files into .c C language files.
.LI
The Unix Shellscript
.I ce-
.br
The shellscript
.I ce
is the compiler command invoked by the user.  It parses the command line
options and source files, applies the appropriate version of the parser
and runtime library, and calls the C compiler cc on the result.
.LI
The ClassC/Elaine runtime routines,
.I libce.a.
.br
libce.a contains runtime functions such as _new(), _deref(), etc.
.EL
.H 1 "Implementation Plan"
.H 2 "Deliverable Items"
.AL
.LI
Compiler-
.br
Working ClassC compiler, runtime routines, and source code.
.LI
Programming Guide-
.br
A description of the language and how to use it.
.LI
Implementation Guide-
.br
A description of how the compiler works and what it does.
.LI
Project Report-
.br
A discussion of the history, goals, and results of the project.
.EL
.H 2 "Completion Criteria"
The ClassC/Elaine language project shall be considered complete when
code written in the ClassC language as specified above can be successfully
compiled and executed.  Two programs, "ChessGen" and "BackPropigation", will
be written in ClassC code to test the functionality of the language.
"ChessGen" generates legal chess moves for specified chess pieces on
a chess board (it does not
.I play
chess).  "BackPropigation" simulates a neural learning network.  These two
simulations were chosen because both game pieces and neurons typify the
sort of abstract entities for which object oriented languages are suited.
In particular, these programs will test the following features of ClassC/Elaine:

.AL
.LI
Use of class member functions and data.
.LI
Redefinition of member functions in derived classes.
.LI
Multiple Inheritance.
.LI
Statement/Declaration mixing.
.LI
Structure and Array components of objects.
.LI
Multiple levels of object dereference (i.e., obj1:>obj2:>obj3:> ... )

.LE

.H 2 "Completion Schedule"
The ClassC/Elaine project will be complete by 31 May 88.
.H 1 "Qualifications"
.H 2 "Professional"
I worked for Hughes Aircraft for four years on computer hardware and
software projects, including two years on assignment in South East Asia.
.H 2 "Academic"
I have no undergraduate computer science education.  At RIT I have taken:
.AL
.LI
ICSS 709	Programming Language Theory
.LI
ICSS 720	Computer Architecture
.LI
ICSS 745	Data Communication and Networks II
.LI
ICSS 706	Foundations of Computing Theory
.LI
ICSS 781	Introduction to Artificial Intelligence
.LI
ICSS 809	Operating Systems I
.LI
ICSS 810	Operating Systems II
.LI
ICSS 850	Computability
.LI
ICSS 851	Computational Complexity
.LI
ICSS 890	Seminar - Speech Processing
.LI
ICSS 899	Independent Study - Design of Activity Coordinator
.EL
.H 2 "Books and Articles"
-- See bibliography (following) --
.H 2 "Previous Investigation"
I have examined the concepts involved in this project and have written
a prototype.
.H 1 "Grading Criteria"
The compiler should be evaluated on how closely it succeeds in implementing
the described language design.  The other deliverable items (Programming
Guide, etc.) should be graded on their clarity and completeness.
.H 1 "Bibliography"

.AL
.LI "[AHO86]"
Aho,A.; Sethi,R.; Ullman,J. [1986]
.I "Compilers - Principles, Techniques, and Tools"
Reading, MA: Addison-Wesley
.LI "[BUD87]"
Budd, Timothy [1987],
.I "A Little Smalltalk",
Reading, MA: Addison-Wesley
.LI "[COX86]"
Cox, Brad [1986],
.I "Object Oriented Programming - An Evolutionary Approach"
Reading, MA: Addison-Wesley
.LI "[GOL83]"
Goldberg,Adele; Robson, David [1983]
.I "Smalltalk-80: The Language and its Implementation"
Reading, MA: Addison-Wesley
.LI "[HAR84]"
Harbison, Samuel; Steele, Guy [1984],
.I "C: A Reference Manual"
Englewood Cliffs, NJ: Prentice Hall, Inc.
.LI "[PRA84]"
Pratt, Terrence [1984]
.I "Programming Languages - Design and Implementation"
Englewood Cliffs, NJ: Prentice Hall, Inc.
.LI "[STR86]"
Stroustrup, Bjarne [1986],
.I "The C++ Programming Language,"
Reading, MA: Addison-Wesley
.TC

An object variable may be declared by a class name,
in which case it must represent only objects of that class or descendants
of that class, or an object variable may be declared by the keyword
.I object,
in which case it can be used to represent any type of object in
the system with no type checking.

