\section{Introduction}
Object oriented programming is a way of abstracting information and
operations to make programming more efficient and reliable.
{\bf C}
is a non object oriented programming language that has become a {\em de facto}
language standard in academic and industrial applications because of its
power and flexibility.  ClassC is an attempt to add object orientation
on the existing framework that C provides.

\subsection{ClassC/Elaine}
{\em ClassC} is a C-based object oriented programming language.  The language
adds object oriented capabilities to the C programming language.
Specifically, ClassC adds the new data type
{\bf object}
, and the new aggregate declaration,
{\bf class}.
Class definitions are like C
structures except they may contain function declarations, and they may
inherit components from multiple ancestors; objects are instances of
a particular class.  An object variable may be declared by a class name,
in which case it must represent only objects of that class or descendants
of that class, or an object variable may be declared by the keyword
{\bf object},
in which case it can be used to represent any type of object in
the system with no type checking.

{\em Elaine}
is the programming environment for ClassC.  Initially it
consists of the runtime library functions necessary to implement
ClassC; enhancements will be added over time.

There are other C-based object oriented languages, such as C++ and
Objective-C, but ClassC differs from these other languages in three points.
\begin{enumerate}
  \item ClassC provides true multiple inheritance, which is very commonly
discussed in the description of object oriented languages, but very
rarely actually implemented.
  \item ClassC provides both strict type checking on objects
as well as typeless dynamic binding --- a variable of type
{\em object} may be assigned any class instantiation.  Dynamic binding is
typical of interpreted
languages like Lisp and Smalltalk, but rare in compiled languages like
C and Pascal.
  \item ClassC offers automatic garbage collection of memory no longer
referenced by any object variables.
\end{enumerate}

\subsection{Motivation}
Object oriented programming provides a way to abstract data and
control.  It allows the programmer to create models that are internally
complex and sophisticated, yet which present a simple and uniform interface
to the user.

There are two broad arguments for Object Oriented
programming.

\begin{enumerate}
\item Object Oriented programming
improves productivity and reliability by allowing the reuse of code and the
standardization of common data types and operations.

\item By allowing greater abstraction, object
oriented languages are continuing the trend in programming languages
towards a more human like and less machine like means of expressing
concepts.  This implies that object oriented programming can facilitate
not only bigger programs, but cleverer ones as well.  This is my primary
interest.
\end{enumerate}

The C language was designed with a particular philosophy as well; and
one which often conflicts with many of the common arguments for object
oriented programming.  In general, C is designed for efficiency and
enforces fairly casual type checking.  C allows the programmer unusual
intimacy with the data representation and the machine.  There is no
information hiding with C.

These characteristics give the programmer flexibility and the ability to
coerce the language in unusual ways.  This is valuable, but can also be
dangerous for unsophisticated users.

Object oriented languages emphasize much more information hiding and
abstraction.  The user is isolated from the machine and data
representation as much as possible.

Both these philosophies have merit.  The task of combining these
features into a single language is to some degree a subjective one, and
one which always makes some compromise for every decision made.

For example, in most object oriented languages, the data components
can not be accessed directly.  The user can only change the data
component through a member function ( or ``message'').  This preserves
abstraction, but is less efficient, more time consuming to write,
and irritating.  The choice made for ClassC was to allow the programmer
direct access to data components of objects.

The goal of ClassC/Elaine, then, is to provide a usable C based object
oriented language that offers an alternative to previous implementations.
Most of the code executed in any program is straight forward and does not
benefit from an object oriented approach ( e.g, for-loops, arithmetic
operations, etc).  This code should be implemented as efficiently as
possible for rapid execution.  This is the motivation for using C as the
base language.  Object type data manipulations should be used for operations
which are conceptually intricate, where use of object oriented techniques
could give both the programmer and the program more clarity and
expressiveness.  This is the motivation for enhancing C with object oriented
features; moreover, this is the reason for giving ClassC a higher level of
abstraction that C++ or ObjectiveC through multiple inheritance
and untyped objects.

In order to be genuinely useful, ClassC was made compatible with existing
C language facilities.  It is compatible with all existing standard
{\bf \#include}
files, as well as all library packages.  It allows separate
compilation and linking of ClassC modules, as well as the creation of
individual ClassC libraries.

\subsection{Background}
The foundations of Object Oriented programming were laid by the
programming language Simula, which was developed in Norway in the early
1960's [DAH66].  It introduced the concept of user defined classes, and was
developed as a simulation language for modeling complex systems.
The power of this concept was not widely recognized until Alan Kay and the
research group at Xerox PARC developed the Smalltalk interactive
environment.  It evolved from Dr. Kay's concept of a computer called the
Dynabook, which would be very user friendly and suitable as a learning
aid for children [GOL83].
Smalltalk remains to this day the prototypical example of
an object oriented language.  It is a rich and complex environment that
shows the power of the object oriented concept.

Yet this richness and complexity has a price.  Smalltalk is big, slow,
expensive, and available on very few machines.  Its primary drawback from a
systems development perspective is its slowness.  Smalltalk runs on a
virtual machine / interpreter, which is inappropriate for large software
systems.  This is unfortunate, because software systems are classic
candidates for the productivity benefits of object oriented programming.
Concepts like queues, stacks, hash tables, etc. are ideal abstraction
types.  But an operating system written in Smalltalk running on conventional
hardware would be very slow.

Enter C++ [STR86].
C++ was developed at AT\&T Bell Laboratories, home of Unix, C, Kernighan and
Ritchie.  It was developed by Bjarne Stroustrup with the intent of giving as
much object oriented structure to C as possible, while sacrificing none of
the efficiency of pure C code.  It is therefore perhaps the best system
development language available today.

The C++ compiler translates C++ code into standard C.  Thus, C++ must implement
its object oriented features in a C format.  It does this by using
standard C structures.  An object in C++ is implemented in two ways.
The function components of the class are given unique names and written
to the output file.  The function members are associated with the class
rather than with the object.  This means that all instances of a class
have unique private data but share the same component functions.

The data portion of a C++ object is implemented in a standard C
structure, with all the data components of the object as elements of
the generated structure.  Inheritance is implemented by
copying the parent object structure and concatenating
the new elements of the derived class after the elements of the parent
class.  The component sequence of the parent and derived classes are therefore
identical up to the end of the parent class.  Consequently, the address
of an instance of a derived class can be assigned to an object pointer
of a parent class and the standard C structure dereferencing operations
for the parent will correctly dereference the derived object
components.
This makes object dereference easy and efficient, but makes multiple
inheritance difficult for reasons discussed later.


Objective-C is a language developed by Dr. Brad Cox at Productivity
Products International and seems much like
C++ with a different syntactic structure.
The two languages offer similar features [COX86].

\subsection{System Description}

The ClassC/Elaine compiler was be developed under Unix using the Unix
lexical analyzer
{\em Lex} and the Unix parser generator {\em Yacc},
and uses the standard Unix/C libraries.  The running system consists
of three files-
\begin{description}
\item [The parser {\em efront---}]
The ClassC/Elaine front end parser, written in C.  The parser translates
ClassC/Elaine .e source files into ``.c'' C language files.
\item[ The Unix Shellscript {\em ce ---}]
The shellscript {\em ce}
is the compiler command invoked by the user.  It parses the command line
options and source files, applies the appropriate version of the parser
and runtime library, and calls the C compiler
{\bf cc} on the result.
\item[The ClassC/Elaine runtime routines, {\em libce.a}]
{\bf libce.a} contains runtime functions such as \_new(), \_deref(), etc.
\end{description}

\subsection{Language Description}

The concept for ClassC differs from those of other C-based object oriented
languages.  ClassC was developed as a
learning tool for data abstraction and object oriented programming
concepts.

Consequently, it emphasizes object oriented functionality at the sacrifice of
some efficiency.  This philosophy is reflected in the decision to provide
multiple inheritance and dynamic binding of objects.  There is some runtime
cost to provide these features, mainly function call overhead.

ClassC is a superset of C; that is, all C operations are supported by
ClassC, and pre-existing library and object files as well as standard
header files are all compatible with ClassC.
The language was developed in a Unix operating system environment,
using the Unix language development tools.  The parser accepts
standard C code and passes it to the output file unchanged.
It maintains symbol tables and scoping information while building the
parse tree.

In addition to standard C, the parser accepts and processes the following:
\begin{description}
\item[ The aggregate definition {\tt class ---}]
A class declaration is much like a standard C structure declaration.
A class is composed of {\em Members};
of which there are two types:
\begin{itemize}
\item {\bf Variable} members ---
which are passive data components of the class instances.
\item {\bf Function} members ---
which are functions that act on the private member variables of class instances.
\end{itemize}
In addition,
a class definition may contain a list of parent classes from which
components are inherited.
\item[The class Member Function definition ---]
Different classes may have member functions with the same name; e.g.,
{\em print}.
To distinguish them, class member functions are defined using the class name
and a scoping operator; e.g., {\tt class\_name::member\_function();} .
\item[Objects ---]
Objects are instances of classes and are dynamically allocated at runtime
by explicit calls to a {\bf new()} function.  A variable of type {\em object}
is really an integer index into an object table.
An object may be declared in one of two ways:

\begin{description}
\item[With a class name ---]
Class names are used to declare object variables.  Object variables declared
this way can only represent objects of that particular
class type or its descendants.
\item[With the type specifier {\tt object} ---]
A variable declared with the type specifier
{\em object}
can be used to represent any type of object instance at all.
This gives greater flexibility, but eliminates any compile time checking of
message validity or type.
\end{description}
\item[The object member dereference ---]
The member dereferencing operator used in ClassC is ``:$>$''.
C++ uses the ordinary C structure dereferencing operators for object
dereferencing, but the dereferencing mechanism in ClassC is implemented
differently and suggests the use of a separate symbol.
\item[ Variable declarations ---]
Variable declarations can be made anywhere in a block.
This is intended to more closely associate the variable declaration
with its use.  Every variable must still
be declared
{\em before} its first use.
\end{description}

\subsection{Functional Overview}

{\bf Classes:}\\
A class is declared as follows:
\begin{verbatim}

class <Class_name>  : <SuperClass_1>  : <SuperClass_2> : ...
	{	<type> <declarator>;
	 	<type> <declarator>;
	 	<type> <declarator>;
	 	<type> <declarator>;
			.
			.
			.
	};



\end{verbatim}
where {\em SuperClass}es
are optional and separated by the inheritance operator ``:''; and where
{\em declarator}
is any type of legal ClassC declarator, including a function
or a class object.

Two special types of
functions that may be declared as a class member are
{\bf init()} and {\bf delete()}.

{\bf init()}
is that class's initializer function.  The init() function is defined by
the user and is automatically invoked when an
object of that class is created.  Otherwise, if init() is not declared,
uninitialized memory space will be allocated.

{\bf delete()}
is the class delete function.  It should be defined by the user to clean up
any special memory resources which objects of that class may allocate to
themselves (through {\em malloc()},
for instance).

{\flushleft \bf Objects:}\\
Objects are created dynamically from the heap; there are no static objects.
Each object is an instance of a specific class.
A member of an object (the member can be either variable or function)
is dereferenced by the object dereferencing operator ``:$>$''.  The general
form of the object dereference is:
\begin{quote} \tt
{\em $<$objectExpression$>$} :> {\em $<$memberName$>$};
\end{quote}
where {\em objectExpression} is usually just a variable name.  In general,
however, an object expression can also be the result of a pointer
dereference, a return value from a function, etc.
The type of the entity returned by the object dereferencing operator is
determined by the type of the object expression.  If the class to which
the object
expression belongs can be determined at compile time, the
type of the returned dereferenced element is determined by its type as
declared in the class definition.

If, however, the object expression evaluates to an instance of the
generic class
{\bf object},
its true class cannot be determined at compile time.  In this
case the type of the result of the dereference is defaults to
{\bf object}
as well.  The programmer can cast the result to be of
any type and override the default.

{\flushleft \bf Member Functions:}\\
Member functions are defined with the class name, scoping operator 
``::'', and the
function name --- e.g, ``Son::print()''.   Arguments are declared just like in
an ordinary C function definition.  An object member function can reference
the other
member variables or functions
of that same object with no special dereferencing or casting
required.  A reference to ``self'' is passed automatically on a call to a
member function, and may be used explicitly by the programmer
in an object's member function definition to
reference the object itself.



Examples:

Declare class {\tt Son} to be a descendant of classes
{\tt Papa} and {\tt Mama}.

\begin{verbatim}

	class Son : Papa : Mama
		{ int a;
		  char * b;
		  init();	/* Optional initialization function */
		  void print();
		};

\end{verbatim}

Define member function init() of class Son:

\begin{verbatim}

	Son::init(x,y)	int x; char * y;
	{ b = (char *) malloc(strlen(y) + 1);
	  strcpy (b,y);
	  a = x;
	}

\end{verbatim}

Declare an object \& initialize to be of class Son:

\begin{verbatim}

	Son xmpl = Son:>new(5,"Hello");

\end{verbatim}


Send the message ``print'' to xmpl:

\begin{verbatim}
	xmpl:>print();
\end{verbatim}

Reset member element ``b'' of xmpl:

\begin{verbatim}
	xmpl:>b = "Goodbye"
\end{verbatim}
%\clearpage
